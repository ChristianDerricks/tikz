% Venn diagram with differently colored parts
% Author: Christian Derricks

% This is tikz implementation for a beautiful Venn diagramm
% The insperation for this came from the Venn diagram in Mathematica
% see: https://www.wolfram.com/language/12/improved-visualization-labeling/venn-diagram.html?product=mathematica
% Circle position and size, colors and label position and size can be controlled by variables

\documentclass[border=2pt]{standalone}%
\usepackage{tikz}%

\usetikzlibrary{calc}%
\usepackage{ifthen}%      used in lines 138 - 157 
\usepackage{anyfontsize}% used in line 49
\usepackage{xcolor}%      used in lines 24 - 31

% show some circles and lines that helped with the construction
\newboolean{showconstructionhelp}%
\setboolean{showconstructionhelp}{false}%

% This is a list of eight colors that are distinguishable for people with 
% every type of color vision deficiency (often misleadingly called colorblindness).
% see https://jfly.iam.u-tokyo.ac.jp/color/ for more information.
\definecolor{cb-darkgrey}{RGB}      { 50,  50,  50}% darkgrey
\definecolor{cb-blue}{RGB}          {  0, 114, 178}% blue
\definecolor{cb-orange}{RGB}        {230, 159,   0}% orange
\definecolor{cb-bluishgreen}{RGB}   {  0, 158, 115}% bluishgreen
\definecolor{cb-vermillion}{RGB}    {213,  94,   0}% vermillion
\definecolor{cb-skyblue}{RGB}       { 86, 180, 233}% skyblue
\definecolor{cb-reddishpurple}{RGB} {204, 121, 167}% reddishpurple
\definecolor{cb-yellow}{RGB}        {240, 228,  66}% yellow

\begin{document}%

% distance between center of circle A and B. 
% Label positions have to be set accordingly.
\pgfmathsetmacro{\a}{2}%

% height of circle C
\pgfmathsetmacro{\h}{(tan(60)/2)*\a}%

% label position of AB, BC and AC with radial symmetry
\pgfmathsetmacro{\r}{1.5}%

% label position of A, B and C with radial symmetry
\pgfmathsetmacro{\R}{1}%

% set the font size of all labels with the anyfontsize package
\tikzstyle{every node}=[font=\fontsize{10}{5}\selectfont]%

% position of a rectanlge that is needed for the multiple clipping
\def\rec{({-\a},{-\a}) rectangle ({2*\a},{\h+\a})}%

% position of all circles
\newcommand{\A}{                   (0,0) circle  (\a)}%
\newcommand{\B}{                  (\a,0) circle  (\a)}%
\newcommand{\C}{               (\a/2,\h) circle  (\a)}%
\newcommand{\X}{(\a/2, {tan(30)*(\a/2)}) circle (0.1)}%

% set colors for all parts
\newcommand{\colorA}{cb-blue}%
\newcommand{\colorB}{cb-orange}%
\newcommand{\colorC}{cb-vermillion}%
\newcommand{\colorAB}{\colorA!50!\colorB!50}%
\newcommand{\colorAC}{\colorC!50!\colorA!50}%
\newcommand{\colorBC}{\colorC!50!\colorB!50}%
\newcommand{\colorABC}{cb-bluishgreen}%
\newcommand{\colorOuterRing}{white}%

\begin{tikzpicture}%

% The geometric shape of all parts of the Venn diagram is defined
% here including the label positions. For the ease of selecting 
% single parts there is no use of loops or lists.

	% A
	\begin{scope}
		\clip \B \rec;
		\clip \C \rec;
		\fill [\colorA!70] \A;
	\end{scope}
	\node at ({0-\R*cos(45)},{0-\R*sin(45)}) {A};

	% B
	\begin{scope}
		\clip \A \rec;
		\clip \C \rec;
		\fill [\colorB!70] \B;
	\end{scope}
	\node at ({\a+\R*cos(45)},{0-\R*sin(45)}) {B};

	% C
	\begin{scope}
		\clip \A \rec;
		\clip \B \rec;
		\fill [\colorC!70] \C;
	\end{scope}
	\node at ({\a/2},{\h+\R}) {C};

	% AB
	\begin{scope}
		\clip \C \rec;
		\clip \B;
		\fill [\colorAB] \A;
	\end{scope}
	\node at (\a/2, {tan(30)*(\a/2)-\r} ) {AB};%

	% AC
	\begin{scope}
		\clip \B \rec;
		\clip \A;
		\fill [\colorAC] \C;
	\end{scope}
	\node at ({\a/2-\r*cos(30)}, {tan(30)*(\a/2)+\r*sin(30)}) {AC};

	% BC
	\begin{scope}
		\clip \A \rec;
		\clip \C;
		\fill [\colorBC] \B;
	\end{scope}
	\node at ({\a/2+\r*cos(30)}, {tan(30)*(\a/2)+\r*sin(30)} ) {BC};

	% ABC
	\begin{scope}
		\clip \A;
		\clip \B;
		\fill [\colorABC!80] \C;
	\end{scope}
	\node at (\a/2, {tan(30)*(\a/2)}) {ABC};

	% white outer rings
	\draw [\colorOuterRing, line width = \a] \A;
	\draw [\colorOuterRing, line width = \a] \B;
	\draw [\colorOuterRing, line width = \a] \C;

	% visible after setting the boolean variable <showconstructionhelp> to true
	\ifshowconstructionhelp
		\draw ({-\a},{-\a}) rectangle ({2*\a},{\h+\a});
		\draw \X;
		\draw (0,0) circle (1);
		\draw (\a,0) circle (1);
		\draw (\a/2,\h) circle (1);
	
		\node at (0,0) {OA};
		\node at (\a,0) {OB};
		\node at (\a/2,\h) {OC};
	
		\draw (0,0) -- (\a,0);
		\draw (\a,0) -- (\a/2,\h);
		\draw (0,0) -- (\a/2,\h);
	
		\draw (\a/2, {tan(30)*(\a/2)}) circle (\r);
		\draw (\a/2, {tan(30)*(\a/2)-\r} ) circle (0.1);
		\draw ({\a/2+\r*cos(30)}, {tan(30)*(\a/2)+\r*sin(30)}) circle (0.1);
		\draw ({\a/2-\r*cos(30)}, {tan(30)*(\a/2)+\r*sin(30)}) circle (0.1);
	\fi

\end{tikzpicture}%

\end{document}%
